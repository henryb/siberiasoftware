\section{Overall Description}

\subsection{Product Perspective}
Students have requested a product to that allows users to play a game of chess at remote locations over the internet, anonymously. AnonChess is one solution to this request, and aims to provide a simple, and reliable way to do so.
\subsubsection{System Interfaces}
\begin{description}
\item[Network] AnonChess utilizes the internet to pass data between the connected parties, there is no direct user contact with the network.
\end{description}
\subsubsection{Hardware Interfaces}
AnonChess runs on any computer that meets the following requirements: 
\begin{itemize}
\item{Capable of running Java SE 6}
\item{Capable of connecting to the Internet}
\item{Has a keyboard and mouse}
\end{itemize}
\subsubsection{Software Interfaces}
 While the application itself is standalone, it is programmed using Java; as such it requires a Java virtual machine to run. The application communicates with a XMPP server to facilitate networking, as explained in subsection 2.1.4.
\subsubsection{Communication Interfaces}
AnonChess uses the XMPP protocol to resolve and connect users; this protocol is both commonly used and has been shown to be reliable.
\subsubsection{Memory Contraints}
AnonChess requires at least 256 MB of memory to run.
\subsubsection{Site Adaptation Requirements}
There are no site adaptation requirements to use AnonChess, however to host AnonChess's servers there are some adaptations required: an XMPP server, and the resolver bot will need to be hosted.

\subsection{User Classes and Characteristics}
The main users of this program will be people who are familiar with the game of chess and wish to play it in an online environment. The opponent's identity is unnecessary and is not required as the software randomly chooses an opponent from the list of connected, available users.

\subsection{Assumptions and Dependencies}
It is assumed that user has sufficient means to input data and receive output from the computer, an internet connection faster than 56kbps (dial-up) and a Java virtual machine.
